% Created 2019-09-10 Tue 10:40
% Intended LaTeX compiler: pdflatex
\documentclass[11pt]{article}
\usepackage[utf8]{inputenc}
\usepackage[T1]{fontenc}
\usepackage{graphicx}
\usepackage{grffile}
\usepackage{longtable}
\usepackage{wrapfig}
\usepackage{rotating}
\usepackage[normalem]{ulem}
\usepackage{amsmath}
\usepackage{textcomp}
\usepackage{amssymb}
\usepackage{capt-of}
\usepackage{hyperref}
\usepackage[top=1in, bottom=1in, left=1in, right=1in]{geometry}
\author{Zach del Rosario (zdelrosario@citrine.io)}
\date{\today}
\title{Lesson Plan: Principles of Visualization}
\hypersetup{
 pdfauthor={Zach del Rosario (zdelrosario@citrine.io)},
 pdftitle={Lesson Plan: Principles of Visualization},
 pdfkeywords={},
 pdfsubject={},
 pdfcreator={Emacs 26.2 (Org mode 9.1.9)},
 pdflang={English}}
\begin{document}

\maketitle

\section{Overview}
\label{sec:org7719cc4}
The purpose of this lesson is to help students understand the fundamental
principles of visualizing data. This lesson draws on the work of Cleveland and
McGill (1985, \emph{Science}) on the \emph{visual hierarchy}, a quantitative ranking of
different methods of encoding numeric values in a visual fashion. As a result
of working through this lesson, students will be able to:

\begin{enumerate}
\item Understand that visualization is accomplished through different \emph{visual
encoding methods} -- the rungs on the visual hierarchy
\item Understand that different rungs on the hierarchy convey information more or
less accurately
\item Use (2) to judge the efficacy of visualizations
\end{enumerate}

\section{Outline and Timings}
\label{sec:org477b319}
\begin{center}
\begin{tabular}{lr}
Task & Min\\
\hline
Setup \& Introduction & 5\\
\hline
P1: \textbf{Card sorting} & \\
Q1 & 5\\
Q2 & 2\\
Q3 & 5\\
Survey & 2\\
Wrap & 5\\
 & (19)\\
\hline
P2: \textbf{Gapminder} & \\
Gapminder A & 7\\
Gapminder B & 4\\
A vs B & 5\\
1 vs 2: walkthrough & 1\\
 & (17)\\
\hline
Wrap & 5\\
(Total) & (46)\\
\end{tabular}
\end{center}

\section{Resources}
\label{sec:org65390cc}
\begin{itemize}
\item \href{https://github.com/zdelrosario/teaching-eda/blob/master/viz\_tasks/challenge.md}{vis-tasks} challenge document
\item \href{https://github.com/zdelrosario/teaching-eda/blob/master/viz\_tasks/grid.pdf}{vis-tasks} printable sheet
\item \href{https://citrineinformatics.github.io/ga-tech-workshop/04-vis-principles/index.html}{vis-tasks notes} with Gapminder examples
\item \href{https://ils.unc.edu/courses/2015\_spring/inls541\_001/Readings/Cleveland\%20and\%20McGill\%201985\%20-\%20Graphical\%20Perception\%20and\%20Cleveland1985-Graphical\%20Methods\%20for\%20Analyzing\%20Scientific\%20Data.pdf}{Cleveland and McGill} (1985) Science
\item \href{https://www.gapminder.org/tools/\#\$chart-type=bubbles}{Gapminder data vis}
\end{itemize}

Table for \textbf{P1: Card sorting}
\begin{itemize}
\item You will iteratively build the following table during Part 1:
\end{itemize}

\begin{center}
\begin{tabular}{rrrll}
1 & 2 & 3 & CODE & Description\\
\hline
 &  &  & COM & Position along a common scale\\
 &  &  & NON & Position along a non-aligned scale\\
 &  &  & LEN & Length\\
 &  &  & ANG & Angle\\
 &  &  & PIR2 & Area\\
 &  &  & SAT & Color saturation\\
\end{tabular}
\end{center}

\section{Part 1: Card sorting}
\label{sec:org78f8136}
\uline{Before:} Do the following:
\begin{itemize}
\item Prepare to show the following websites (open in tabs):
\begin{itemize}
\item \href{https://citrineinformatics.github.io/ga-tech-workshop/05-vis-principles/index.html}{Pages site} -- Main tool for presenting
\item \href{http://ils.unc.edu/courses/2015\_spring/inls541\_001/Readings/Cleveland\%20and\%20McGill\%201985\%20-\%20Graphical\%20Perception\%20and\%20Cleveland1985-Graphical\%20Methods\%20for\%20Analyzing\%20Scientific\%20Data.pdf}{Cleveland and McGill (1985)} -- reference for the visual hierarchy
\item \href{https://www.gapminder.org/tools/\#\$chart-type=bubbles}{Gapminder visualization} -- to show how the Gapminder project shows their own data
\end{itemize}
\item Hand out cards for pairs
\end{itemize}

\uline{Beginning:} Say the following
\begin{itemize}
\item "For the next 50 minutes, we are going to work through a guided exercise
together. You are going to work together in pairs on a set of questions, and
we will share ideas in a full group discussin."
\item "The cards you have depict a dataset. These data are from a dataset on
imported cars; the `Count` is the number of vehicles of particular types in
the dataset. We are going to focus mainly on how the data are displayed."
\item "For the moment, we are going to focus on \uline{the same, single variable},
displayed in different ways."
\item (Show Q1 through Q3 on the \href{https://citrineinformatics.github.io/ga-tech-workshop/05-vis-principles/index.html}{Pages site}. \textbf{Zoom the page for visibility}.)
\end{itemize}

\uline{Q1:} (5 Minutes) How is `Count` displayed?
\begin{itemize}
\item Give one answer: "Position along a common scale"
\item (While students are working, write the six codes stacked vertically on the
board in order, i.e. the CODE column in the Table above.)
\begin{itemize}
\item COM, NON, LEN, ANG, PIR2, SAT
\end{itemize}
\item (Walk among students, listen in on their thinking, ask for volunteers to share
in the larger group. Aim for 2-3 volunteers.)
\item (Bring students back together. Get volunteers to share their ideas.)
\item (Fill in the full table to disambiguate codenames:)
\begin{itemize}
\item COM -- Common scale
\item NON -- Non-aligned scale
\item LEN -- Length
\item ANG -- Angle
\item PIR2 -- Area
\item SAT -- Color saturation
\end{itemize}
\end{itemize}

\uline{Q2:} (2 Minutes) Answer the questions:
\begin{itemize}
\item "Which is larger? The `Count` of `wagons` with `fwd drive` OR the `Count` of
`wagons` with `rwd drive`?
(Count of `fwd wagon` is \textbf{larger} than `rwd wagon`)
\item "By how much is one larger than the other?" (By about 2-3 counts.)
\item (Make sure the \href{https://citrineinformatics.github.io/ga-tech-workshop/05-vis-principles/index.html}{Pages site} is visible to students; it is difficult to keep
these questions in one's head!)
\item (Can skip walking among the students; this question is very quick.)
\end{itemize}

\uline{Q3:} (5 Minutes) Rank the six visualizations \emph{in terms of how well they help you answer
Q2}. Rate the most helpful visualization 1, and the least helpful as 6; do not
allow ties between visualizations. \textbf{Make sure to rank at least your top 3.}
\begin{itemize}
\item (While students are working, fill in the grid lines from rank 1 to 3. Complete
the table.)
\item (Walk among students, listen in on their thinking, ask why they chose the
particular rankings they did.)
\end{itemize}

\uline{Survey:} (2 Minutes) How many people Ranked COM 1? Ranked COM 2? Ranked COM 3?, etc. until
pattern emerges.
\begin{itemize}
\item (Bring students back together. Count hands \textbf{for each cell} in the top three
rows. There should be a nearly-diagonal pattern.)
\item Ask "Why are we seeing this pattern of preference? It has to do with the
\emph{visual hierarchy}."
\end{itemize}

\uline{Cleveland and McGill (1985)} (5 Minutes)
\begin{itemize}
\item "These visualizations are based on the rungs of Cleveland and McGill's \emph{visual
hierarchy}. They studied \uline{how accurately} people could interpret graphs, based
on how the data were visually encoded. They arrived at the following
hierarchy, in descending order of accuracy:"
\begin{enumerate}
\item Position along a common scale
\item Position on identical but nonaligned scales
\item Length
\item Angle; Slope (With slope not too close to 0, 90, or 180 degrees)
\item Area
\item Volume; Density; Color saturation
\item Color hue
\end{enumerate}
\begin{itemize}
\item Use \href{https://citrineinformatics.github.io/ga-tech-workshop/05-vis-principles/index.html}{Pages site} to quickly show hierarchy
\end{itemize}
\item (An aside: We asked something slightly different -- preference rather than accuracy.)
\item \href{https://ils.unc.edu/courses/2015\_spring/inls541\_001/Readings/Cleveland\%20and\%20McGill\%201985\%20-\%20Graphical\%20Perception\%20and\%20Cleveland1985-Graphical\%20Methods\%20for\%20Analyzing\%20Scientific\%20Data.pdf}{publication link}

\item "We can \emph{use} this insight to help judge and design graphs."
\end{itemize}

\section{Part 2: Gapminder}
\label{sec:org4fc256a}
\begin{itemize}
\item Send remote folks to \href{https://citrineinformatics.github.io/ga-tech-workshop/04-vis-principles/index.html}{online notes}
\item "Now, we are going to consider \uline{four different variables}, displayed on the
same graph in different ways.
\item "The data are from the Gapminder project, which seeks to educate people about
global poverty."
\end{itemize}

\textbf{Gapminder A vs B}
\begin{itemize}
\item \uline{Q1:} (7 Minutes) Gapminder A
\begin{itemize}
\item (2 Minutes) How are the four variables `Population, GDP per Capita, Life Expectancy at
Birth, Continent` encoded visually?
\begin{itemize}
\item (Have students Think-Pair-Share)
\end{itemize}
\item (5 Minutes) What observations can you make about the data based on the vis?
\begin{itemize}
\item (Have students Think-Pair-Share)
\end{itemize}
\end{itemize}

\item \uline{Q2:} (4 Minutes) Gapminder B
\begin{itemize}
\item (2 Minutes) How are the four variables `Population, GDP per Capita, Life Expectancy at
Birth, Continent` encoded visually?
\begin{itemize}
\item (Talk through this; ask for volunteers on the spot.)
\end{itemize}
\item (2 Minutes) What observations can you make about the data based on the vis?
\end{itemize}

\item \textbf{A vs B} (5 Minutes)
\begin{itemize}
\item "I find it easier to see the lower life expectancy in `Africa` based on Gapminder B."
\item "With the \emph{visual hierarchy}, we can be more specific than 'this graph is
bad' -- we can note that 'Continent has fewer levels, therefore it is easier
to show with a color scale.'"
\item Same data, same variables, different choice of encoding
\end{itemize}
\end{itemize}

\textbf{Timeseries 1 vs 2} (1 Minute)
\begin{itemize}
\item Walkthrough two different visualizations
\item "Here the choice is less obvious; depends on which variable I care about more
-- GDP / capita or Life expectancy."
\item "\textbf{Visualization is an iterative process.}"
\end{itemize}

\section{Finale}
\label{sec:orgf35abf9}
\begin{itemize}
\item Wrapup (5 Minutes) "What did we talk about?"

\item "Variables/numbers are \emph{encoded} visually to make a graph; we choose how to encode"
\begin{itemize}
\item (Gesture to the table to remind)
\end{itemize}
\item "We need to make these decisions when constructing a visualization."
\begin{itemize}
\item (Scroll through the Gapminder examples, use them to illustrate different choices.)
\end{itemize}
\item Lessons:
\begin{itemize}
\item \textbf{Use our linear scales preferentially} -- reserve for the "most important"
continuous variables
\item \textbf{Use secondary scales strategically} -- use for "secondary" variables, or
categorical variables
\end{itemize}
\item \textbf{Visualization is an iterative process}
\begin{itemize}
\item We won't get the design perfect the first time.
\end{itemize}
\begin{itemize}
\item What is "good" depends on our goals; exploratory graphs have very different
goals than communication graphs.
\end{itemize}
\end{itemize}
\end{document}
